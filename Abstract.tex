Emotion recognition in conversations is a growing area of research in the field of affective computing. Historically, various methods have been proposed for emotion inference in conversations including rule-based, machine learning-based, and deep learning-based approaches. In recent years, the use of pre-trained language models and transfer learning has been increasingly popular as a means of enhancing the capacities of emotion recognition in conversations. Another promising approach is the integration of common sense knowledge with emotion recognition models. Additionally, context and sentiment-aware graph networks have also been proposed as a method of improving the performance of these models. This literature review will contrast and compare these methods, discussing their limitations and advantages in order to provide a comprehensive overview of the current state of the field.